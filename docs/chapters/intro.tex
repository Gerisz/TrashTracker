\chapter{Bevezetés}
\label{ch:intro}

Az illegális szemétlerakatok jelensége modern világunkban egy súlyos probléma, amely nem csak környezetünket károsítja, hanem egészségünkre és közösségeink jólétére is kihathat. Ez a jelenség fokozottan megfigyelhető Magyarországon is, főként folyóink és tavaink, illetve településink határaiban. Ahogy világunk fejlődik és változik, úgy egyre több, ember által okozott szennyezéssel kell szembenéznünk azáltal, hogy megőrizzük környezetünk tisztaságát és egészségét. Habár ehhez az első lépés az emberek figyelmének felhívása, viszont a második mindenképpen a már természetbe jutott szemét mihamarabbi eltávolítása és nem utolsó sorban a környezet rehabilitálása.\par
Ezen dolgozatnak a célja is hasonló, amely az előbb említett illegális szemétlerakatok adatbázisba gyűjtése a lakosság segítségével, illetve annak térképes vizualizációja, mely segít azoknak a civil önkénteseknek, akik a környezetvédelem érdekében alkalmilag, vagy akár rendszeresen takarítják és szállítják el ezeket. Ehhez társul egy minimalista, közösségi médiához hasonló felület is, ahol minden regisztrált felhasználó a saját profilján gyűjtheti a már általa bejelentett pontokat. A célt és az implementációt a \textit{Tiszta Tisza Térkép} ihlette, ahol a \textit{TrashOut} nevű alkalmazásból nyert végpontokat tüntetik fel hasonlóan a weboldalukon, mely a weboldal tulajdonosai által évente megrendezett \textit{PET Kupa} című hulladékgyűjtő verseny alkalmával használják azt lehetséges szemétlelőhelyek fellelésére. Míg az előbb említett oldalon csak a pontok megtekintésére van lehetőség, az előbb említett alkalmazásból kinyert adatokkal, addig ebben a programban lehetőség van jóváhagyásos regisztrálás után azok bejelentésére, illetve azok adatainak frissítésére/javítására is.
\newpage