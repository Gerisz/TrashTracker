\chapter{Fejlesztői dokumentáció}
\label{ch:impl}

\section{Megoldási terv}

\subsection{Követelményelemzés}

A weboldal használatakor négy típusú felhasználó különböztethető meg: vendég, bejelentkezett felhasználó, moderátor és adminisztrátor. Az előző fejezethez hasonlóan, itt is ezek alapján történik meg a funkcionális követelmények definiálása.

\subsubsection{Vendég}
\begin{compactitem}
	\item Térkép megtekintése:
		\begin{compactitem}
			\item Alapvető irányítási lehetőségek: mozgatás, forgatás, nagyítás/kicsinyítés
			\item Az összes pont helyzetének megfelelő megjelenítése, fontosabb információk jelölésével
			\item Pontra kattintáskor sorszám, kép és megjegyzés megtekintése, hivatkozás annak adatlapjára
		\end{compactitem}
	\item Pont részletezése:
		\begin{compactitem}
			\item Minden információ megjelenítése
			\item Közvetlen hivatkozások a feltöltő profiljára, a térképen való megtekintésre és az adatlap megtekintésére
		\end{compactitem}
	\item Pontok listázása:
		\begin{compactitem}
			\item Pontok közötti szűrés, lapméret állítása
			\item Fontosabb információk megjelenítése
			\item Közvetlen hivatkozások a feltöltő profiljára, a térképen való megtekintésre és az adatlap megtekintésére
		\end{compactitem}
	\item Színséma választása:
		\begin{compactitem}
			\item Sötét és világos mód közötti váltás bárhol 
		\end{compactitem}
	\item Regisztráció:
	\begin{compactitem}
		\item Felhasználónév megadása
		\item E-mail cím megadása
		\item Jelszó megadása és megerősítése
		\item Profilkép megadása
		\item Regisztráció gombra nyomva:
		\begin{compactitem}
			\item siker esetén: automatikus bejelentkeztetés az új fiókba
			\item sikertelenség esetén (hiányzik egy kötelező adat, vagy felhasználónév/e-mail cím már foglalt): hibaüzenet megjelenítése a sikertelenség okával feltüntetve
		\end{compactitem}
	\end{compactitem}
	\item Bejelentkezés:
	\begin{compactitem}
		\item Felhasználónév megadása
		\item Jelszó megadása
		\item Bejelentkezés gombra nyomva:
		\begin{compactitem}
			\item siker esetén: bejelentkeztetés a fiókba
			\item sikertelenség esetén (hiányzik egy adat, vagy felhasználó-jelszó páros nem megfelelő): hibaüzenet megjelenítése a sikertelenség okát NEM feltüntetve
		\end{compactitem}
	\end{compactitem}
\end{compactitem}

\subsubsection{Bejelentkezett felhasználó}
\begin{compactitem}
	\item Minden, amit egy vendég is tud (értelemszerűen a regisztráció és bejelentkezés kivételével)
	\item Kijelentkezés:
	\begin{compactitem}
		\item Kijelentkezés gombra kattintva kijelentkezteti a felhasználót és elnavigálja a kezdőoldalra
	\end{compactitem}
	\item Új pont bejelentése:
	\begin{compactitem}
		\item Koordináták (kötelező) megadása, saját koordináták automatikus meghatározása
		\item Település és -rész megadása
		\item Méret megadása
		\item Hozzáférhetőség megadása
		\item Szemét típusának megadása
		\item Opcionális megjegyzés megadása
		\item Képek hozzáadása
		\item Bejelentés gombra nyomva:
		\begin{compactitem}
			\item siker esetén: pont létrehozása és elnavigálás az előzőleg látogatott aloldalra
			\item sikertelenség esetén (hiányzó koordináták, érvénytelen adatok megadása): hibaüzenet megjelenítése a sikertelenség okát feltüntetve
		\end{compactitem}
	\end{compactitem}
	\item Pont szerkesztése:
	\begin{compactitem}
		\item Minden bejelentéskor megadott adat frissítése
		\item Új állapot megadása
		\item Módosítás gombra nyomva: hasonló funkcionalitás a bejelentésekor kattintott Bejelentés gomb esetén
	\end{compactitem}
	\item Saját profil megtekintése:
	\begin{compactitem}
		\item Profilkép megjelenítése
		\item Felhasználónév megjelenítése
		\item E-mail cím megjelenítése
		\item Hivatkozás saját profil szerkesztésére
		\item Saját bejelentett pontok listázása (hasonló módon az összes pont listázásához)
	\end{compactitem}
	\item Saját profil szerkesztése:
	\begin{compactitem}
		\item Minden regisztráláskor megadott adat frissítése
		\item Módosítás gombra nyomáskor:
		\begin{compactitem}
			\item siker esetén: felhasználó adatainak módosítása és elnavigálás az előzőleg látogatott aloldalra
			\item sikertelenség esetén (foglalt felhasználónév vagy e-mail cím, érvénytelen kép megadása): hibaüzenet megjelenítése a sikertelenség okát feltüntetve
		\end{compactitem}
	\end{compactitem}
\end{compactitem}

\subsubsection{Moderátor}
\begin{compactitem}
	\item Minden, amit egy bejelentkezett felhasználó (ezzel együtt vendég) is tud
	\item Pont törlése
	\begin{compactitem}
		\item Megerősítő üzenet a törlésről, igen/nem válasszal:
		\begin{compactitem}
			\item igen esetén: pont végleges törlése, ezzel többé nem megjelenítve a térképen, sem a listanézeten
			\item nem esetén: visszanavigálás az előzőleg látogatott aloldalra
		\end{compactitem}
	\end{compactitem}
\end{compactitem}

\subsubsection{Adminisztrátor}
\begin{compactitem}
	\item Minden, amit egy moderátor is tud
	\item Felhasználói fiókok listázása:
	\begin{compactitem}
		\item Felhasználónév megjelenítése
		\item E-mail cím megjelenítése
		\item Regisztrálás idejének megjelenítése
		\item Hivatkozások az adott fiók profiljára, módosítására és törlésére
	\end{compactitem}
	\item Felhasználói fiókok módosítása:
	\begin{compactitem}
		\item Saját profil szerkesztésekor megadható adatok megadása
		\item Új szerepkör megadása
		\item Módosítás gombra nyomáskor: a saját profil szerkesztéséhez hasonló viselkedés
	\end{compactitem}
	\item Felhasználói fiókok törlése:
	\begin{compactitem}
		\item Megerősítő üzenet a törlésről, igen/nem válasszal:
		\begin{compactitem}
			\item igen esetén: a fiók végleges törlése, ezzel többé elérhetetlenné téve a fiókot, ha be van jelentkezve, akkor kijelentkeztetésre kerül azonnal
			\item nem esetén: visszanavigálás az előzőleg látogatott aloldalra
		\end{compactitem}
	\end{compactitem}
\end{compactitem}

\subsection{Rendszer-architektúra}

A weboldal elsősorban C\# nyelven, az ASP.NET Core keretrendszer használatával ún. MVC (Model-View-Controller, azaz Modell-Nézet-Vezérlő) architektúrában van megvalósítva. Másodsorban JavaScript nyelven is, egyes oldalak kliensoldali dinamikusságáért, túlnyomó részt a térképes megjelenítés esetében. Maguk, a statikus oldalak tartalmai C\# segítségével összeállított HTML oldalakból állnak, melynek stílusbeli megjelenéséért a CSS stílusleíró nyelvben írt Bootstrap 5.3 keretrendszer felel.\\
A felhasználó egy webböngészőben létsít kapcsolatot az MVC architektúra vezérlő-rétegével, mely segítségével a model-réteg bizonyos műveleteit meghívva az adatokban állapotváltozást ér el. Ezen változásokat a nézet-réteg közvetlen, vagy közvetve egy nézetmodell (vagy DTO, azaz Data Transfer Object, magyarul adatátviteli objektum) segítségével nézetet (jelen esetben egy weboldalt) szolgáltat vissza.\\
Egy ilyen vezérlőhöz érkező kéréskor a program példányosítja a megfelelő vezérlő-osztályt és az ahhoz szükséges egyéb objektumokat, majd a nézet elkészítése és visszaküldése után ezek megsemmisítésre kerülnek. Emellett előfordul, hogy nem egy nézet lesz a válasz, hanem egy akció eredmény, mely kiegészülhet egy eredménnyel.

\section{Rétegek}

\subsection{Perzisztencia-réteg}
\label{subsec:persistence}

Ennek a rétegnek, mint neve is sugallja, az adatok tárolása a célja, mely nem hagyja elveszni azokat a program leállítása esetén. Ehhez egy MSSQL-adatbázis áll a rendelkezésre, melyet az alkalmazás implementációjához használt Entity Framework Core (röviden EF) objektum-relációs leképzőrendszer hoz létre ún. migráció alkalmával, illetve létesít kapcsolatot a program és az adatbázis között. Ez a rendszer a programban definiált osztályokat felelteti meg az adatbázis tábláinak oly módon, hogy egy osztály tulajdonsága lesz a relációs tábla egy-egy oszlopa (vagy más néven attribútuma), illetve az osztály példányai (azaz objektumai) az adatbázis egy-egy sora (vagy más néven rekordja).\\
Az adatbázis felépítése code-first megközelítéssel történik, ami azt jelenti, hogy a program C\# nyelvű forráskódjából áll elő az adatbázis. A leképzőrendszer nem csak az osztályok tulajdonságait veszi figyelembe, hanem azoknak adattípusait, illetve egyedi megszorításait is megfelelteti MSSQL-oldalon. Emellett felismeri az osztályok tulajdonságainak neve alapján a táblák közötti relációkat is (egy az egyhez, egy a többhöz és több a többhöz kapcsolatokat).\\
Emellett ilyen kapcsolatok esetén lehetőség van ún. navigációs tulajdonságok deklarálására is, mely az idegen kulcs forráskód oldali használata helyett közvetlen a kapcsolatban álló objektumra mutat (melyek értelemszerűen nem jelennek meg az adatbázisban, ott továbbra is csak az idegen kulcs található meg), ezzel megkönnyítve annak fejlesztését. A legjobb teljesítmény érdekében ezek a navigációs tulajdonságok lusta kiértékeléssel vannak ellátva, mely azt jelenti, hogy csak akkor lesz beletöltve valós érték, amikor arra szükség van és/vagy explicit kérjük azokat. Amíg ezek nem kerülnek betöltésre, addig ezek helyén egy helyettes (idegen szóval proxy) tervezési minta által készített leszármaztatott proxy lesz.\\
Azt, hogy mely osztályok definiálnak egy-egy táblát, azt egy adatbázis-kontextus osztályban kell felsorolni, melyen keresztül lehetőség lesz az adatbázissal való kommunikációra is. Emellett nem csak a táblák felsorolása, hanem az azokra vonatkozó megszorítások is itt szerepel. A programkódban történő adatmódosítások perzisztálását külön eljárás meghívásával kell elérni, mely tulajdonképpen a követett módosításokat menti az adatbázisba.\\
Nem csak táblákat, hanem lekérdezéseket is lehet C\# kódról SQL-re fordítani, mivel a programnyelv támogatja az abba beágyazott lekérdezéseket (LINQ-et) is. Ilyen lekérdezések a kontextuson való meghívásával a rendszer automatikusan SQL-re fordítva hajtja végre azokat.

\subsubsection{Adatbázis}

Az előző \ref{subsec:persistence} alszakaszban tárgyaltak szerint az adatbázis a \texttt{TrashTracker.Data.Models} névtérben található \texttt{TrashTrackerDbContext} adatbázis-kontextus osztály alapján definiált táblákból áll össze. Az ebben az osztályban \texttt{DbSet} generikus tulajdonságokkal felsorolt táblák az alábbiak:
\begin{compactitem}
	\item \texttt{Trashes} (típusa \texttt{DbSet<Trash>}): a bejelentett pontokat, illetve azok információit tartalmazza
	\item \texttt{TrashImages} (típusa \texttt{DbSet<TrashImage>}): a bejelentett pontokhoz tartozó (akár több) képeket tartalmazza, egy \texttt{Trashes} rekordhoz több \texttt{TrashImages} rekord tartozhat
	\item \texttt{UserImages} (típusa \texttt{DbSet<UserImage>}): a felhasználókhoz tartozó profilképet, illetve azokhoz társuló egyéb adatokat tartalmazza
\end{compactitem}
Az adatbázis-kontextus osztály az ASP.NET Core \texttt{IdentityDbContext} osztályából származik le, melynek típusparaméterei a \texttt{TrashTrackerUser} (ez definiálja a felhasználók attribútumait), illetve a \texttt{TrashTrackerIdentityRole} (ez definiálja a szerepkörök adatait) osztályok, így további táblák kerülnek létrehozásra a felhasználó- és szerepkörkezelés céljára:
\begin{compactitem}
	\item \texttt{AspNetUsers} tábla, mely az alapértelmezett tulajdonságok (pl. felhasználónév, e-mail cím és jelszó) mellett \texttt{TrashTrackerUser} osztály által definiált tulajdonságokat egybefoglalva tárolja az összes regisztrált felhasználó adatait. Egy az egyhez kapcsolatban áll a \texttt{UserImages} táblával, mivel egy felhasználónak egyszerre csak egy profilképe lehet.
	\item \texttt{AspNetRoles} tábla az összes, jelen esetben felhasználó, moderátor és adminisztrátor szerepköröket tárolja.
	\item \texttt{AspNetUserRoles} tábla több a többhöz kapcsolatot alakít ki a felhasználók és a hozzájuk tartozó szerepkörök között, egy-egy felhasználó (-kulcs) és szerepkör (-kulcs) páros segítségével.
	\item \texttt{AspNetUserTokens} tábla, mely a bejelentkezés-kezelőnek biztosít perzisztenciát a felhasználókhoz társított ún. tokenekkel, melyekkel bejelentkezés után a felhasználónak nem kell minden aloldal látogatás során újra és újra bejelentkeznie, hanem ezzel igazolja, hogy valóban ő az, aki bejelentkezett.
\end{compactitem}
\begin{note}
	A fentebbi felsorolásban csak a program szempontjából releváns táblák lettek felsorolva. Az ASP.NET Core több táblát is definiál (pl. felhasználókhoz vagy szerepkörökhöz jogokhoz \texttt{AspNetUserClaims} és \texttt{AspNetRoleClaims} táblákat), viszont ezek nincsenek felhasználva a program működése során.
\end{note}

\begin{figure}[H]
	\centering
	\subcaptionbox{Az adatbázis osztályai forráskód oldalon}{
		\includegraphics[width=0.7\textwidth]{database_cs}}
	\vspace{5pt}
	\subcaptionbox{Az adatbázis sémája SQL oldalon}{
		\includegraphics[width=0.7\linewidth]{database_sql}}
	\caption{Az adatbázis sémája leképzés előtt és után}
	\label{fig:database_scheme}
\end{figure}

\subsection{DTO-réteg}

A \texttt{TrashTracker.Data.DTOs} névtérben találhatóak az ún. DTO-k, magyarul adatátviteli osztályok. Ezeknek célja az adatbázisban található összes adat helyett csak az adott kontextusban szükséges adatokra való projektálás (pl. listanézet esetén nincs szükség a képekre, ezzel jelentős adatforgalmat megtakarítva). Emellett az egyes adatoknak itt vannak definiálva a validációs kritériumai és -hibaüzenetei (pl. a koordináták -180 és 180 között kell legyenek), ha azok bemeneti adatokként is szolgálnak, ellentétben nincs rá szükség, mivel feltehetjük, hogy a már adatbázisban szerepló adatok lekérdezéskor megfelelnek a megszorításoknak). Az adatátvitelhez használt objektumok az alábbiak:\\
\texttt{TrashTracker.Data.DTOs.In} (ki- és bemenő adatokhoz, validációkkal)
\begin{compactitem}
	\item \texttt{TokenFromGoogle}: a TrashOut-hoz való hozzáféréshez tartalmazza a tokent
	\item \texttt{TrashEdit}: egy bejelentett pont szerkeszthető tulajdonságait tartalmazza, melyből frissíthető egy \texttt{Trash} objektum
	\item \texttt{TrashFromTrashout}: egy TrashOut-ról letöltött pont adatait tartalmazza, melyből előállítható egy \texttt{Trash} objektum
	\item \texttt{TrashFromUser}: egy felhasználó által bejelentett pont adatait tartalmazza, melyből előállítható egy \texttt{Trash} objektum
	\item \texttt{UserEdit}: egy felhasználói profil szerkeszthető adatait tartalmazza, melyből frissíthető egy \texttt{TrashTrackerUser} objektum
	\item \texttt{UserLogin}: egy felhasználó bejelentkezéshez szükséges felhasználónév (vagy e-mail cím) és jelszó párosát tartalmazza
	\item \texttt{UserRegister}: egy felhasználó profil regisztráláshoz szükséges/megadható adatait tartalmazza, melyből előállítható egy \texttt{TrashTrackerUser} objektum
\end{compactitem}
\texttt{TrashTracker.Data.DTOs.Out} (kizárólag kimenő adatokhoz, validációk nélkül)
\begin{compactitem}
	\item \texttt{TrashDetails}
	\item \texttt{TrashIndex}
	\item \texttt{TrashMap}
	\item \texttt{TrashMapDetails}
	\item \texttt{UserDetails}
	\item \texttt{UserIndex}
\end{compactitem}