\chapter{Manuális teszteredmények}
\label{appx:simulation}

\section{Vendég}
\label{sec:sim_guest}

\begin{center}
	\centering
	\begin{longtable}{ | m{0.025\textwidth}| m{0.1\textwidth} | m{0.75\textwidth} | }
		\hline
		\multicolumn{3}{ | c | }{\textbf{Navigációs sáv} (\ref{subsec:nav_guest} alszakasz)} \\
		\hline
		\multicolumn{2}{ | l | }{AS A} & vendég \\
		\hline
		\multicolumn{2}{ | l | }{I WANT TO} & valamelyik aloldalra navigálni \\
		\hline
		\multicolumn{2}{ | l | }{SO THAT} & azt az aloldalt elérjem \\
		\hline
		\multirow{3}{*}{1} & GIVEN & bármely oldalon vagyok \\
		\cline{2-3}
		& WHEN & a navigációs sávban a "Térkép" feliratra kattintok \\
		\cline{2-3}
		& THEN & a főoldalra/térképre leszek navigálva \\
		\hline
		\multirow{3}{*}{2} & GIVEN & bármely oldalon vagyok \\
		\cline{2-3}
		& WHEN & a navigációs sávban a "Lista" feliratra kattintok \\
		\cline{2-3}
		& THEN & a pontok listanézetére leszek navigálva \\
		\hline
		\multirow{3}{*}{3} & GIVEN & bármely oldalon vagyok \\
		\cline{2-3}
		& WHEN & a navigációs sávban a "Nincs bejelentkezve" feliratra kattintok \\
		\cline{2-3}
		& THEN & lenyílik egy almenü, benne "Bejelentkezés" és "Regisztráció" feliratokkal \\
		\hline
		\multirow{3}{*}{4} & GIVEN & bármely oldalon vagyok, a "Nincs bejelentkezve" feliratra kattintva \\
		\cline{2-3}
		& WHEN & a lenyíló almenüben a "Bejelentkezés" feliratra kattintok \\
		\cline{2-3}
		& THEN & a bejelentkezési űrlapra leszek navigálva \\
		\hline
		\multirow{3}{*}{5} & GIVEN & bármely oldalon vagyok, a "Nincs bejelentkezve" feliratra kattintva \\
		\cline{2-3}
		& WHEN & a lenyíló almenüben a "Regisztráció" feliratra kattintok \\
		\cline{2-3}
		& THEN & a regisztrációs űrlapra leszek navigálva \\
		\hline
		\multirow{3}{*}{6} & GIVEN & bármely oldalon vagyok \\
		\cline{2-3}
		& WHEN & a jobb oldalon található színséma kapcsolóra kattintok \\
		\cline{2-3}
		& THEN & a weboldalak színsémája átvált sötét és világos között a másikba \\
		\hline
		\multirow{3}{*}{7} & GIVEN & bármely oldalon vagyok, nem az alapértelmezett (sötét) színsémával \\
		\cline{2-3}
		& WHEN & átnavigálok bármely aloldalra \\
		\cline{2-3}
		& THEN & a színséma megmarad az aloldalon is \\
		\hline
	\end{longtable}
	\label{tab:sim_guest_navigation}
\end{center}

\begin{center}
	\centering
	\begin{longtable}{ | m{0.025\textwidth}| m{0.1\textwidth} | m{0.75\textwidth} | }
		\hline
		\multicolumn{3}{ | c | }{\textbf{Térkép} (\ref{subsec:map_guest} alszakasz)} \\
		\hline
		\multicolumn{2}{ | l | }{AS A} & vendég (vagy felhasználó, moderátor, adminisztrátor) \\
		\hline
		\multicolumn{2}{ | l | }{I WANT TO} & térképen böngészni \\
		\hline
		\multicolumn{2}{ | l | }{SO THAT} & a térképet böngésszem \\
		\hline
		\multirow{3}{*}{1} & GIVEN & a főoldalon/térképen vagyok egy asztali számítógépen \\
		\cline{2-3}
		& WHEN & a térképre bal egérgombot húzva \\
		\cline{2-3}
		& THEN & a térképet mozgatom \\
		\hline
		\multirow{3}{*}{2} & GIVEN & a főoldalon/térképen vagyok egy asztali számítógépen \\
		\cline{2-3}
		& WHEN & a térképre egérgörgőt felfelé, illetve lefelé görgetve \\
		\cline{2-3}
		& THEN & a térképet nagyítom, illetve kicsinyítem \\
		\hline
		\multirow{3}{*}{3} & GIVEN & a főoldalon/térképen vagyok egy asztali számítógépen \\
		\cline{2-3}
		& WHEN & a térképre Shift billentyű nyomása mellett jobb egérgombot húzva \\
		\cline{2-3}
		& THEN & a térképet nagyítom/kicsinyítem és forgatom, az egérgomb lehúzásának kezdeti pozíciójához mérten \\
		\hline
		\multirow{3}{*}{4} & GIVEN & a főoldalon/térképen vagyok egy érintőképernyős eszközön \\
		\cline{2-3}
		& WHEN & a térképre nyomva ujjat húzva \\
		\cline{2-3}
		& THEN & a térképet mozgatom \\
		\hline
		\multirow{3}{*}{5} & GIVEN & a főoldalon/térképen vagyok egy érintőképernyős eszközön \\
		\cline{2-3}
		& WHEN & a térképre nyomva két ujjat széttolva, illetve összehúzva \\
		\cline{2-3}
		& THEN & a térképet nagyítom, illetve kicsinyítem \\
		\hline
		\multirow{3}{*}{6} & GIVEN & a főoldalon/térképen vagyok \\
		\cline{2-3}
		& WHEN & a térkép bal alsó sarkában található \faIcon{filter} gombra kattintva \\
		\cline{2-3}
		& THEN & a szűrési menü beúszik alulról \\
		\hline
		\multirow{3}{*}{7} & GIVEN & a főoldalon/térképen vagyok, a szűrési menüt megnyitva \\
		\cline{2-3}
		& WHEN & a menü bármely tulajdonságot ábrázoló gombjára kattintva \\
		\cline{2-3}
		& THEN & annak logikai állapota átvált és a térképen található pontok azoknak megfelelően szűrődnek \\
		\hline
		\multirow{3}{*}{8} & GIVEN & a főoldalon/térképen vagyok, a szűrési menüt megnyitva \\
		\cline{2-3}
		& WHEN & a menü jobb felső sarkában található $\times$ gombra kattintva \\
		\cline{2-3}
		& THEN & a szűrési menü bezárul \\
		\hline
		\multirow{3}{*}{9} & GIVEN & a főoldalon/térképen vagyok \\
		\cline{2-3}
		& WHEN & a térkép jobb alsó sarkában található \faIcon{download} gombra kattintva \\
		\cline{2-3}
		& THEN & a pontok .csv-be szedett formátumban fájlként letöltésre kerülnek \\
		\hline
		\multirow{3}{*}{10} & GIVEN & a főoldalon/térképen vagyok \\
		\cline{2-3}
		& WHEN & a térképen megjelenített pontra kattintva \\
		\cline{2-3}
		& THEN & a ponton egy modális buborékban megjelenik annak sorszáma, egy képe és megjegyzése, egy hivatkozással annak adatlapjára \\
		\hline
		\multirow{3}{*}{11} & GIVEN & a főoldalon/térképen vagyok, egy azon található pontra kattintva \\
		\cline{2-3}
		& WHEN & a modális buborékban található "Részletek" gombra kattintok \\
		\cline{2-3}
		& THEN & annak adatlapjára leszek navigálva \\
		\hline
		\multirow{3}{*}{12} & GIVEN & a főoldalon/térképen vagyok, egy azon található pontra kattintva \\
		\cline{2-3}
		& WHEN & a modális buborékra kétszer kattintok \\
		\cline{2-3}
		& THEN & a buborékot bezárom \\
		\hline
	\end{longtable}
	\label{tab:sim_guest_map}
\end{center}

\begin{center}
	\centering
	\begin{longtable}{ | m{0.025\textwidth}| m{0.1\textwidth} | m{0.75\textwidth} | }
		\hline
		\multicolumn{3}{ | c | }{\textbf{Pont részletezése (\ref{subsec:trash_details} alszakasz)}} \\
		\hline
		\multicolumn{2}{ | l | }{AS A} & vendég \\
		\hline
		\multicolumn{2}{ | l | }{I WANT TO} & pont adatlapját megtekinteni \\
		\hline
		\multicolumn{2}{ | l | }{SO THAT} & a pont adatlapján szereplő adatokat megismerni \\
		\hline
		\multirow{3}{*}{1} & GIVEN & a pont adatlapján vagyok \\
		\cline{2-3}
		& WHEN & a pont feltöltőjére vagy forrására kattintok \\
		\cline{2-3}
		& THEN & a pont feltöltőjének profiljára vagy annak forrására leszek navigálva \\
		\hline
		\multirow{3}{*}{2} & GIVEN & a pont adatlapján vagyok \\
		\cline{2-3}
		& WHEN & a pont koordinátáira kattintok \\
		\cline{2-3}
		& THEN & a térképre leszek navigálva, az adott pontra nagyítva \\
		\hline
	\end{longtable}
	\label{tab:sim_guest_trash_details}
\end{center}

\begin{center}
	\centering
	\begin{longtable}{ | m{0.025\textwidth}| m{0.1\textwidth} | m{0.75\textwidth} | }
		\hline
		\multicolumn{3}{ | c | }{\textbf{Pontok listanézete (\ref{subsec:trash_index} alszakasz)}} \\
		\hline
		\multicolumn{2}{ | l | }{AS A} & vendég \\
		\hline
		\multicolumn{2}{ | l | }{I WANT TO} & pontok listanézetét megtekinteni \\
		\hline
		\multicolumn{2}{ | l | }{SO THAT} & a pontok listanézetben böngésszem \\
		\hline
		\multirow{3}{*}{1} & GIVEN & a pont listanézetének oldalán vagyok \\
		\cline{2-3}
		& WHEN & a szűrési feltételeket megadva a "Keresés" gombra kattintok \\
		\cline{2-3}
		& THEN & a listát az azoknak megfelelő pontokkal az ott megadott oldalméretben kapom vissza \\
		\hline
		\multirow{3}{*}{2} & GIVEN & a pont listanézetének oldalán vagyok \\
		\cline{2-3}
		& WHEN & a "Vissza a teljes listához" gombra kattintok \\
		\cline{2-3}
		& THEN & a listát alapértelmezett (100 pont/oldal, meg nem tisztított összes pont) keresési feltételeivel kapom vissza \\
		\hline
		\multirow{3}{*}{3} & GIVEN & a pont listanézetének oldalán vagyok \\
		\cline{2-3}
		& WHEN & egy oldalszámra vagy a \faIcon{angle-double-left}, \faIcon{angle-left}, \faIcon{angle-right}, \faIcon{angle-double-right} ikonok valamelyikére kattintok \\
		\cline{2-3} 
		& THEN & az adott számú, vagy az első, előző, következő, illetve utolsó oldalra leszek navigálva \\
		\hline
		\multirow{3}{*}{4} & GIVEN & a pont listanézetének oldalán vagyok \\
		\cline{2-3}
		& WHEN & az egyik pont (sor) feltöltőjére vagy forrására kattintok \\
		\cline{2-3}
		& THEN & az adott pont feltöltőjének profiljára vagy forrására leszek navigálva \\
		\hline
		\multirow{3}{*}{5} & GIVEN & a pont listanézetének oldalán vagyok \\
		\cline{2-3}
		& WHEN & az egyik pont (sor) koordinátáira kattintok \\
		\cline{2-3}
		& THEN & a térképre leszek navigálva, az adott pontra nagyítva \\
		\hline
		\multirow{3}{*}{6} & GIVEN & a pont listanézetének oldalán vagyok \\
		\cline{2-3}
		& WHEN & az egyik pont (sor) jobb végén található \faIcon{info-circle} gombra kattintok \\
		\cline{2-3}
		& THEN & annak adatlapjára leszek navigálva \\
		\hline
	\end{longtable}
	\label{tab:sim_guest_trash_index}
\end{center}

\begin{center}
	\centering
	\begin{longtable}{ | m{0.025\textwidth}| m{0.1\textwidth} | m{0.75\textwidth} | }
		\hline
		\multicolumn{3}{ | c | }{\textbf{Regisztráció (\ref{subsec:register} alszakasz)}} \\
		\hline
		\multicolumn{2}{ | l | }{AS A} & vendég \\
		\hline
		\multicolumn{2}{ | l | }{I WANT TO} & regisztrálni egy felhasználói fiókot \\
		\hline
		\multicolumn{2}{ | l | }{SO THAT} & hozzáférhessek további funkciókhoz \\
		\hline
		\multirow{3}{*}{1} & GIVEN & a regisztrációs űrlapon vagyok \\
		\cline{2-3}
		& WHEN & a "Bejelentkezés" gombra kattintok \\
		\cline{2-3}
		& THEN & a bejelentkezési űrlapra leszek navigálva \\
		\hline
		\multirow{3}{*}{2} & GIVEN & a regisztrációs űrlapon vagyok \\
		\cline{2-3}
		& WHEN & az adatokat helyesen töltöttem ki és a "Regisztráció" gombra kattintok \\
		\cline{2-3}
		& THEN & a kezdőoldalra leszek navigálva, az új fiókommal automatikusan bejelentkeztetve \\
		\hline
		\multirow{3}{*}{3} & GIVEN & a regisztrációs űrlapon vagyok \\
		\cline{2-3}
		& WHEN & felhasználónevet üresen hagyva, vagy már használtat választva a "Regisztráció" gombra kattintok \\
		\cline{2-3}
		& THEN & újra a regisztrációs oldalra leszek navigálva, az eddig kitöltött adatokkal és a releváns hibaüzenetekkel \\
		\hline
		\multirow{3}{*}{4} & GIVEN & a regisztrációs űrlapon vagyok \\
		\cline{2-3}
		& WHEN & az e-mail címet üresen hagyva, nem megfelelő formátumú (azaz nem e-mail címet), vagy már használtat választva a "Regisztráció" gombra kattintok \\
		\cline{2-3} 
		& THEN & újra a regisztrációs oldalra leszek navigálva, az eddig kitöltött adatokkal és a releváns hibaüzenetekkel \\
		\hline
		\multirow{3}{*}{5} & GIVEN & a regisztrációs űrlapon vagyok \\
		\cline{2-3}
		& WHEN & a jelszót üresen hagyva, vagy nem megfelelő formátumút (azaz kevesebb, mint 6 karakter, illetve nem tartalmaz legalább egy kis- és nagybetűt, illetve számot) választva, a "Regisztráció" gombra kattintok \\
		\cline{2-3} 
		& THEN & újra a regisztrációs oldalra leszek navigálva, az eddig kitöltött adatokkal és a releváns hibaüzenetekkel \\
		\hline
		\multirow{3}{*}{6} & GIVEN & a regisztrációs űrlapon vagyok \\
		\cline{2-3}
		& WHEN & a jelszó megerősítését nem a jelszóval megegyező adattal kitöltve, a "Regisztráció" gombra kattintok \\
		\cline{2-3} 
		& THEN & újra a regisztrációs oldalra leszek navigálva, az eddig kitöltött adatokkal és a releváns hibaüzenetekkel \\
		\hline
		\multirow{3}{*}{7} & GIVEN & a regisztrációs űrlapon vagyok \\
		\cline{2-3}
		& WHEN & a "Fájl kiválasztása" gombra kattintok \\
		\cline{2-3} 
		& THEN & megnyílik egy fájlválasztós ablak \\
		\hline
		\multirow{3}{*}{8} & GIVEN & a regisztrációs űrlapon vagyok, a "Fájl kiválasztása" gombbal megnyitott fájlválasztós ablakkal\\
		\cline{2-3}
		& WHEN & kiválasztok egy fájlt, amely kép és .jpg (vagy .jpeg) vagy .png formátumú, majd az ablak "Megnyitás" gombjára kattintok \\
		\cline{2-3} 
		& THEN & bezárul az ablak és beillesztésre kerül a fájl a "Profilkép" mezőbe \\
		\hline
		\multirow{3}{*}{9} & GIVEN & a regisztrációs űrlapon vagyok, a "Fájl kiválasztása" gombbal megnyitott fájlválasztós ablakkal\\
		\cline{2-3}
		& WHEN & kiválasztok egy fájlt, amely nem .jpg (vagy .jpeg) vagy .png formátumú, majd az ablak "Megnyitás" gombjára kattintása után a "Regisztráció" gombra kattintok \\
		\cline{2-3} 
		& THEN & bezárul az ablak és beillesztésre kerül a fájl a "Profilkép" mezőbe, majd újra a regisztrációs oldalra leszek navigálva, az eddig kitöltött adatokkal és a releváns hibaüzenetekkel \\
		\hline
	\end{longtable}
	\label{tab:sim_register}
\end{center}

\begin{center}
	\centering
	\begin{longtable}{ | m{0.025\textwidth}| m{0.1\textwidth} | m{0.75\textwidth} | }
		\hline
		\multicolumn{3}{ | c | }{\textbf{Bejelentkezés (\ref{subsec:login} alszakasz)}} \\
		\hline
		\multicolumn{2}{ | l | }{AS A} & vendég \\
		\hline
		\multicolumn{2}{ | l | }{I WANT TO} & bejelentkezni a saját fiókomba \\
		\hline
		\multicolumn{2}{ | l | }{SO THAT} & hozzáférni adataimhoz és funkcióihoz \\
		\hline
		\multirow{3}{*}{1} & GIVEN & a bejelentkezési űrlapon vagyok \\
		\cline{2-3}
		& WHEN & a "Regisztráció" gombra kattintok \\
		\cline{2-3}
		& THEN & a regisztrációs űrlapra leszek navigálva \\
		\hline
		\multirow{3}{*}{2} & GIVEN & a bejelentkezési űrlapon vagyok \\
		\cline{2-3}
		& WHEN & az adatokat helyesen töltöttem ki és a "Bejelentkezés" gombra kattintok \\
		\cline{2-3}
		& THEN & a kezdőoldalra leszek navigálva, a megadott adatokkal egyező fiókkal bejelentkeztetve \\
		\hline
		\multirow{3}{*}{3} & GIVEN & a bejelentkezési űrlapon vagyok \\
		\cline{2-3}
		& WHEN & felhasználónevet üresen hagyva, a "Bejelentkezés" gombra kattintok \\
		\cline{2-3}
		& THEN & újra a bejelentkezési oldalra leszek navigálva, az eddig kitöltött adatokkal és egy általános hibaüzenettel \\
		\hline
		\multirow{3}{*}{4} & GIVEN & a bejelentkezési űrlapon vagyok \\
		\cline{2-3}
		& WHEN & a jelszót üresen hagyva, a "Bejelentkezés" gombra kattintok \\
		\cline{2-3} 
		& THEN & újra a bejelentkezési oldalra leszek navigálva, az eddig kitöltött adatokkal és egy általános hibaüzenettel \\
		\hline
		\end{longtable}
	\label{tab:sim_login}
\end{center}\newpage

\section{Bejelentkezett felhasználó}
\label{sec:sim_user}

Minden, amik az előző, \ref{sec:sim_guest} szakaszban felsorolt felhasználói történetek, azok egyaránt vonatkoznak a nála magasabb hozzáféréssel rendelkező fiókokra, így egy bejelentkezett felhasználóra is. Ha egy funkció kiegészült további funkciókkal, akkor itt \textbf{csak a kiegészítések szerepelnek}.

\begin{center}
	\centering
	\begin{longtable}{ | m{0.025\textwidth}| m{0.1\textwidth} | m{0.75\textwidth} | }
		\hline
		\multicolumn{3}{ | c | }{\textbf{Pontok listanézete (\ref{subsec:trash_index} alszakasz)}} \\
		\hline
		\multicolumn{2}{ | l | }{AS A} & bejelentkezett felhasználó \\
		\hline
		\multicolumn{2}{ | l | }{I WANT TO} & pontok listanézetét megtekinteni \\
		\hline
		\multicolumn{2}{ | l | }{SO THAT} & a pontokat listanézetben böngésszem, azokat bővítsem és/vagy szerkesszem \\
		\hline
		\multirow{3}{*}{1} & GIVEN & a pont listanézetének oldalán vagyok \\
		\cline{2-3}
		& WHEN & a lista tetejének jobb végén található \faIcon{plus} gombra kattintok \\
		\cline{2-3}
		& THEN & annak szerkesztési űrlapjára leszek navigálva \\
		\hline
		\multirow{3}{*}{2} & GIVEN & a pont listanézetének oldalán vagyok \\
		\cline{2-3}
		& WHEN & az egyik pont (sor) jobb végén található \faIcon{edit} gombra kattintok \\
		\cline{2-3}
		& THEN & annak szerkesztési űrlapjára leszek navigálva \\
		\hline
	\end{longtable}
	\label{tab:sim_user_trash_index}
\end{center}

\begin{center}
	\centering
	\begin{longtable}{ | m{0.025\textwidth}| m{0.1\textwidth} | m{0.75\textwidth} | }
		\hline
		\multicolumn{3}{ | c | }{\textbf{Pont bejelentése (\ref{subsec:trash_create} alszakasz)}} \\
		\hline
		\multicolumn{2}{ | l | }{AS A} & bejelentkezett felhasználó \\
		\hline
		\multicolumn{2}{ | l | }{I WANT TO} & bejelenteni egy illegális szemétlerakatot \\
		\hline
		\multicolumn{2}{ | l | }{SO THAT} & a térképen és listában megjelenjen \\
		\hline
		\multirow{3}{*}{1} & GIVEN & a pont bejelentési űrlapján oldalán vagyok \\
		\cline{2-3}
		& WHEN & az (legalább kötelező) adatokat helyesen kitöltve a "Bejelentés" gombra kattintok \\
		\cline{2-3}
		& THEN & az előző oldalra leszek navigálva, illetve a bejelentett pontom megjelenik a térképen, a listanézeten és a profiloldalamon \\
		\hline
		\multirow{3}{*}{2} & GIVEN & a pont bejelentési űrlapján oldalán vagyok \\
		\cline{2-3}
		& WHEN & a "Vissza" gombra kattintok \\
		\cline{2-3}
		& THEN & az előző oldalra leszek navigálva\\
		\hline
		\multirow{3}{*}{3} & GIVEN & a pont bejelentési űrlapján oldalán vagyok \\
		\cline{2-3}
		& WHEN & legalább az egyik koordinátát nem -180° és 180° között kitöltve a "Bejelentés" gombra kattintok \\
		\cline{2-3}
		& THEN & újra a bejelentési űrlapra leszek navigálva, az eddig kitöltött adatokkal és releváns hibaüzenettel \\
		\hline
		\multirow{3}{*}{4} & GIVEN & a pont bejelentési űrlapján oldalán vagyok \\
		\cline{2-3}
		& WHEN & a megjegyzést több, mint 2000 karakterrel kitöltve a "Bejelentés" gombra kattintok \\
		\cline{2-3}
		& THEN & újra a bejelentési űrlapra leszek navigálva, az eddig kitöltött adatokkal és releváns hibaüzenettel \\
		\hline
		\multirow{3}{*}{5} & GIVEN & a pont bejelentési űrlapján oldalán vagyok \\
		\cline{2-3}
		& WHEN & kiválasztok egy fájlt, amely kép és .jpg (vagy .jpeg) vagy .png formátumú, majd az ablak "Megnyitás" gombjára kattintok \\
		\cline{2-3} 
		& THEN & bezárul az ablak és beillesztésre kerül a fájl a "Képek hozzáadása" mezőbe \\
		\hline
		\multirow{3}{*}{6} & GIVEN & a pont bejelentési űrlapján oldalán vagyok \\
		\cline{2-3}
		& WHEN & kiválasztok egy (vagy több) fájl(oka)t, amely nem .jpg (vagy .jpeg) vagy .png formátumú, majd az ablak "Megnyitás" gombjára kattintása után a "Bejelentés" gombra kattintok \\
		\cline{2-3} 
		& THEN & bezárul az ablak és beillesztésre kerül a fájl a "Képek hozzáadása" mezőbe, majd újra a regisztrációs oldalra leszek navigálva, az eddig kitöltött adatokkal és a releváns hibaüzenetekkel \\
		\hline
	\end{longtable}
	\label{tab:sim_trash_create}
\end{center}

\begin{center}
	\centering
	\begin{longtable}{ | m{0.025\textwidth}| m{0.1\textwidth} | m{0.75\textwidth} | }
		\hline
		\multicolumn{3}{ | c | }{\textbf{Pont szerkesztése (\ref{subsec:trash_edit} alszakasz)}} \\
		\hline
		\multicolumn{2}{ | l | }{AS A} & bejelentkezett felhasználó \\
		\hline
		\multicolumn{2}{ | l | }{I WANT TO} & szerkeszteni egy pontot \\
		\hline
		\multicolumn{2}{ | l | }{SO THAT} & frissítsem annak adatait \\
		\hline
		\multirow{3}{*}{1} & GIVEN & a pont szerkesztési űrlapján oldalán vagyok \\
		\cline{2-3}
		& WHEN & az (legalább kötelező) adatokat helyesen kitöltve a "Módosítás" gombra kattintok \\
		\cline{2-3}
		& THEN & az előző oldalra leszek navigálva, illetve a bejelentett pontom megjelenik a térképen, a listanézeten és a profiloldalamon \\
		\hline
		\multirow{3}{*}{2} & GIVEN & a pont szerkesztési űrlapján oldalán vagyok \\
		\cline{2-3}
		& WHEN & a "Vissza" gombra kattintok \\
		\cline{2-3}
		& THEN & az előző oldalra leszek navigálva\\
		\hline
		\multirow{3}{*}{3} & GIVEN & a pont szerkesztési űrlapján oldalán vagyok \\
		\cline{2-3}
		& WHEN & legalább az egyik koordinátát nem -180° és 180° között kitöltve a "Módosítás" gombra kattintok \\
		\cline{2-3}
		& THEN & újra a bejelentési űrlapra leszek navigálva, az eddig kitöltött adatokkal és releváns hibaüzenettel \\
		\hline
		\multirow{3}{*}{4} & GIVEN & a pont szerkesztési űrlapján oldalán vagyok \\
		\cline{2-3}
		& WHEN & a megjegyzést több, mint 2000 karakterrel kitöltve a "Módosítás" gombra kattintok \\
		\cline{2-3}
		& THEN & újra a bejelentési űrlapra leszek navigálva, az eddig kitöltött adatokkal és releváns hibaüzenettel \\
		\hline
		\multirow{3}{*}{5} & GIVEN & a pont szerkesztési űrlapján oldalán vagyok \\
		\cline{2-3}
		& WHEN & kiválasztok egy fájlt, amely kép és .jpg (vagy .jpeg) vagy .png formátumú, majd az ablak "Megnyitás" gombjára kattintok \\
		\cline{2-3} 
		& THEN & bezárul az ablak és beillesztésre kerül a fájl a "Képek hozzáadása" mezőbe \\
		\hline
		\multirow{3}{*}{6} & GIVEN & a pont szerkesztési űrlapján oldalán vagyok \\
		\cline{2-3}
		& WHEN & kiválasztok egy (vagy több) fájl(oka)t, amely nem .jpg (vagy .jpeg) vagy .png formátumú, majd az ablak "Megnyitás" gombjára kattintása után a "Módosítás" gombra kattintok \\
		\cline{2-3} 
		& THEN & bezárul az ablak és beillesztésre kerül a fájl a "Képek hozzáadása" mezőbe, majd újra a regisztrációs oldalra leszek navigálva, az eddig kitöltött adatokkal és a releváns hibaüzenetekkel \\
		\hline
	\end{longtable}
	\label{tab:sim_trash_edit}
\end{center}

\begin{center}
	\centering
	\begin{longtable}{ | m{0.025\textwidth}| m{0.1\textwidth} | m{0.75\textwidth} | }
		\hline
		\multicolumn{3}{ | c | }{\textbf{Saját (illetve más) profil megtekintése (\ref{subsec:user_details} alszakasz)}} \\
		\hline
		\multicolumn{2}{ | l | }{AS A} & bejelentkezett felhasználó \\
		\hline
		\multicolumn{2}{ | l | }{I WANT TO} & megtekinteni saját vagy más profiloldalát \\
		\hline
		\multicolumn{2}{ | l | }{SO THAT} & lássam a profilon található képet, nevet és elérhetőséget, illetve a bejelentett pontjait \\
		\hline
		\multirow{3}{*}{1} & GIVEN & egy profil oldalán vagyok \\
		\cline{2-3}
		& WHEN & a e-mail cím gombjára kattintok \\
		\cline{2-3}
		& THEN & az alapértelmezett levelező programban megnyílik az adott e-mail címre egy új üzenet írása \\
		\hline
		\multirow{3}{*}{2} & GIVEN & a saját profilom oldalán vagyok \\
		\cline{2-3}
		& WHEN & a "Profil szerkesztése" gombra kattintok \\
		\cline{2-3}
		& THEN & a profil szerkesztésének űrlapjára leszek navigálva \\
		\hline
		\multirow{3}{*}{3} & GIVEN & egy profil oldalán vagyok \\
		\cline{2-3}
		& WHEN & egy oldalszámra vagy a \faIcon{angle-double-left}, \faIcon{angle-left}, \faIcon{angle-right}, \faIcon{angle-double-right} ikonok valamelyikére kattintok \\
		\cline{2-3} 
		& THEN & az adott számú, vagy az első, előző, következő, illetve utolsó oldalra leszek navigálva \\
		\hline
		\multirow{3}{*}{4} & GIVEN & egy profil oldalán vagyok \\
		\cline{2-3}
		& WHEN & az egyik pont (sor) feltöltőjére kattintok \\
		\cline{2-3}
		& THEN & az adott pont feltöltőjének profiljára (azaz ismét ide) leszek navigálva \\
		\hline
		\multirow{3}{*}{5} & GIVEN & egy profil oldalán vagyok \\
		\cline{2-3}
		& WHEN & az egyik pont (sor) koordinátáira kattintok \\
		\cline{2-3}
		& THEN & a térképre leszek navigálva, az adott pontra nagyítva \\
		\hline
		\multirow{3}{*}{6} & GIVEN & egy profil oldalán vagyok \\
		\cline{2-3}
		& WHEN & az egyik pont (sor) jobb végén található \faIcon{info-circle} gombra kattintok \\
		\cline{2-3}
		& THEN & annak adatlapjára leszek navigálva \\
		\hline
		\multirow{3}{*}{7} & GIVEN & egy profil oldalán vagyok \\
		\cline{2-3}
		& WHEN & a lista tetejének jobb végén található \faIcon{plus} gombra kattintok \\
		\cline{2-3}
		& THEN & annak szerkesztési űrlapjára leszek navigálva \\
		\hline
		\multirow{3}{*}{8} & GIVEN & egy profil oldalán vagyok \\
		\cline{2-3}
		& WHEN & az egyik pont (sor) jobb végén található \faIcon{edit} gombra kattintok \\
		\cline{2-3}
		& THEN & annak szerkesztési űrlapjára leszek navigálva \\
		\hline
	\end{longtable}
\label{tab:sim_user_details}
\end{center}

\begin{center}
	\centering
	\begin{longtable}{ | m{0.025\textwidth}| m{0.1\textwidth} | m{0.75\textwidth} | }
		\hline
		\multicolumn{3}{ | c | }{\textbf{Saját profil szerkesztése (\ref{subsec:user_edit} alszakasz)}} \\
		\hline
		\multicolumn{2}{ | l | }{AS A} & bejelentkezett felhasználó, moderátor \\
		\hline
		\multicolumn{2}{ | l | }{I WANT TO} & szerkeszteni a saját profilomat \\
		\hline
		\multicolumn{2}{ | l | }{SO THAT} & frissítsem annak adatait \\
		\hline
		\multirow{3}{*}{1} & GIVEN & a profil szerkesztési űrlapon vagyok \\
		\cline{2-3}
		& WHEN & a "Vissza" gombra kattintok \\
		\cline{2-3}
		& THEN & az előző oldalra leszek navigálva \\
		\hline
		\multirow{3}{*}{2} & GIVEN & a profil szerkesztési űrlapon vagyok \\
		\cline{2-3}
		& WHEN & az adatokat helyesen töltöttem ki és a "Módosítás" gombra kattintok \\
		\cline{2-3}
		& THEN & a előző oldalra leszek navigálva, az fiókom adatait frissítve \\
		\hline
		\multirow{3}{*}{3} & GIVEN & a profil szerkesztési űrlapon vagyok \\
		\cline{2-3}
		& WHEN & felhasználónevet üresen hagyva, vagy már használtat választva a "Módosítás" gombra kattintok \\
		\cline{2-3}
		& THEN & újra a szerkesztési űrlap oldalára leszek navigálva, az eddig kitöltött adatokkal és a releváns hibaüzenetekkel \\
		\hline
		\multirow{3}{*}{4} & GIVEN & a profil szerkesztési űrlapon vagyok \\
		\cline{2-3}
		& WHEN & az e-mail címet üresen hagyva, nem megfelelő formátumú (azaz nem e-mail címet), vagy már használtat választva a "Módosítás" gombra kattintok \\
		\cline{2-3} 
		& THEN & újra a szerkesztési űrlap oldalára leszek navigálva, az eddig kitöltött adatokkal és a releváns hibaüzenetekkel \\
		\hline
		\multirow{3}{*}{5} & GIVEN & a profil szerkesztési űrlapon vagyok \\
		\cline{2-3}
		& WHEN & a "Fájl kiválasztása" gombra kattintok \\
		\cline{2-3} 
		& THEN & megnyílik egy fájlválasztós ablak \\
		\hline
		\multirow{3}{*}{6} & GIVEN & a profil szerkesztési űrlapon vagyok, a "Fájl kiválasztása" gombbal megnyitott fájlválasztós ablakkal \\
		\cline{2-3}
		& WHEN & kiválasztok egy fájlt, amely kép és .jpg (vagy .jpeg) vagy .png formátumú, majd az ablak "Megnyitás" gombjára kattintok \\
		\cline{2-3} 
		& THEN & bezárul az ablak és beillesztésre kerül a fájl a "Profilkép" mezőbe \\
		\hline
		\multirow{3}{*}{7} & GIVEN & a profil szerkesztési űrlapon vagyok, a "Fájl kiválasztása" gombbal megnyitott fájlválasztós ablakkal \\
		\cline{2-3}
		& WHEN & kiválasztok egy fájlt, amely nem .jpg (vagy .jpeg) vagy .png formátumú, majd az ablak "Megnyitás" gombjára kattintása után a "Regisztráció" gombra kattintok \\
		\cline{2-3} 
		& THEN & bezárul az ablak és beillesztésre kerül a fájl a "Profilkép" mezőbe, majd újra a a szerkesztési űrlap oldalára leszek navigálva, az eddig kitöltött adatokkal és a releváns hibaüzenetekkel \\
		\hline
	\end{longtable}
\label{tab:sim_user_edit}
\end{center}

\begin{center}
	\centering
	\begin{longtable}{ | m{0.025\textwidth}| m{0.1\textwidth} | m{0.75\textwidth} | }
		\hline
		\multicolumn{3}{ | c | }{\textbf{Kijelentkezés (\ref{subsec:logout} alszakasz)}} \\
		\hline
		\multicolumn{2}{ | l | }{AS A} & bejelentkezett felhasználó, moderátor, adminisztrátor \\
		\hline
		\multicolumn{2}{ | l | }{I WANT TO} & kijelentkezni a saját fiókomból \\
		\hline
		\multicolumn{2}{ | l | }{SO THAT} & többé ne lehessen hozzáférni adataimhoz és funkcióihoz, egy újabb bejelentkezésig \\
		\hline
		\multirow{3}{*}{1} & GIVEN & bármely oldalon vagyok \\
		\cline{2-3}
		& WHEN & a navigációs sáv jobb szélén található profilnevemre kattintva lenyíló menüben a "Kijelentkezés" pontra kattintok \\
		\cline{2-3}
		& THEN & a főoldalra leszek navigálva és kijelentkeztetésre kerülök \\
		\hline
	\end{longtable}
\label{tab:sim_logout}
\end{center}

\section{Moderátor}
\label{sec:sim_moderator}

Minden, amik az előző, \ref{sec:sim_user} szakaszban felsorolt felhasználói történetek, azok egyaránt vonatkoznak a nála magasabb hozzáféréssel rendelkező fiókokra, így egy moderátorra is. Ha egy funkció kiegészült további funkciókkal, akkor itt \textbf{csak a kiegészítések szerepelnek}.

\begin{center}
	\centering
	\begin{longtable}{ | m{0.025\textwidth}| m{0.1\textwidth} | m{0.75\textwidth} | }
		\hline
		\multicolumn{3}{ | c | }{\textbf{Pontok listanézete (\ref{subsec:trash_index} alszakasz)}} \\
		\hline
		\multicolumn{2}{ | l | }{AS A} & moderátor, adminisztrátor \\
		\hline
		\multicolumn{2}{ | l | }{I WANT TO} & pontok listanézetét megtekinteni \\
		\hline
		\multicolumn{2}{ | l | }{SO THAT} & a pontokat listanézetben böngésszem, azokat bővítsem, szerkesszem és/vagy töröljem \\
		\hline
		\multirow{3}{*}{1} & GIVEN & a fiókok listanézetének oldalán vagyok \\
		\cline{2-3}
		& WHEN & az egyik pont (sor) jobb végén található \faIcon{trash} gombra kattintok \\
		\cline{2-3}
		& THEN & annak törlésének megerősítő üzenetének űrlapjára leszek navigálva \\
		\hline
	\end{longtable}
	\label{tab:sim_moderator_trash_index}
\end{center}

\begin{center}
	\centering
	\begin{longtable}{ | m{0.025\textwidth}| m{0.1\textwidth} | m{0.75\textwidth} | }
		\hline
		\multicolumn{3}{ | c | }{\textbf{Saját (illetve más) profil megtekintése (\ref{subsec:user_details} alszakasz)}} \\
		\hline
		\multicolumn{2}{ | l | }{AS A} & moderátor, adminisztrátor \\
		\hline
		\multicolumn{2}{ | l | }{I WANT TO} & megtekinteni saját vagy más profiloldalát \\
		\hline
		\multicolumn{2}{ | l | }{SO THAT} & lássam a profilon található képet, nevet és elérhetőséget, illetve a bejelentett pontjait \\
		\hline
		\multirow{3}{*}{1} & GIVEN & egy profil oldalán vagyok \\
		\cline{2-3}
		& WHEN & az egyik pont (sor) jobb végén található \faIcon{trash} gombra kattintok \\
		\cline{2-3}
		& THEN & annak törlésének megerősítő üzenetének űrlapjára leszek navigálva \\
		\hline
	\end{longtable}
	\label{tab:sim_moderator_user_details}
\end{center}

\begin{center}
	\centering
	\begin{longtable}{ | m{0.025\textwidth}| m{0.1\textwidth} | m{0.75\textwidth} | }
		\hline
		\multicolumn{3}{ | c | }{\textbf{Pont törlése (\ref{subsec:trash_delete} alszakasz)}} \\
		\hline
		\multicolumn{2}{ | l | }{AS A} & moderátor, adminisztrátor \\
		\hline
		\multicolumn{2}{ | l | }{I WANT TO} & törölni egy pontot \\
		\hline
		\multicolumn{2}{ | l | }{SO THAT} & többé ne jelenjen meg a térképen és listákban \\
		\hline
		\multirow{3}{*}{1} & GIVEN & a pont törlésének megerősítő kérdésének űrlapján vagyok \\
		\cline{2-3}
		& WHEN & a "Törlés" gombra kattintok \\
		\cline{2-3}
		& THEN & az előző oldalra leszek navigálva és a pont törlésre kerül \\
		\hline
		\multirow{3}{*}{2} & GIVEN & a pont törlésének megerősítő kérdésének űrlapján vagyok \\
		\cline{2-3}
		& WHEN & a "Vissza" gombra kattintok \\
		\cline{2-3}
		& THEN & az előző oldalra leszek navigálva \\
		\hline
	\end{longtable}
	\label{tab:sim_trash_delete}
\end{center}

\section{Adminisztrátor}
\label{sec:sim_admin}

Minden, amik az előző, \ref{sec:sim_moderator} szakaszban felsorolt felhasználói történetek, azok egyaránt vonatkoznak a nála magasabb hozzáféréssel rendelkező fiókokra, így egy adminisztrátorra is. Ha egy funkció kiegészült további funkciókkal, akkor itt \textbf{csak a kiegészítések szerepelnek}.

\begin{center}
	\centering
	\begin{longtable}{ | m{0.025\textwidth}| m{0.1\textwidth} | m{0.75\textwidth} | }
		\hline
		\multicolumn{3}{ | c | }{\textbf{Navigációs sáv} (\ref{subsec:nav_guest} alszakasz)} \\
		\hline
		\multicolumn{2}{ | l | }{AS A} & adminisztrátor \\
		\hline
		\multicolumn{2}{ | l | }{I WANT TO} & valamelyik aloldalra navigálni \\
		\hline
		\multicolumn{2}{ | l | }{SO THAT} & azt az aloldalt elérjem \\
		\hline
		\multirow{3}{*}{1} & GIVEN & bármely oldalon vagyok \\
		\cline{2-3}
		& WHEN & a navigációs sávban a "Felhasználók" feliratra kattintok \\
		\cline{2-3}
		& THEN & a felhasználók listanézetére leszek navigálva \\
		\hline
	\end{longtable}
	\label{tab:sim_admin_navigation}
\end{center}

\begin{center}
	\centering
	\begin{longtable}{ | m{0.025\textwidth}| m{0.1\textwidth} | m{0.75\textwidth} | }
		\hline
		\multicolumn{3}{ | c | }{\textbf{Felhasználói fiókok listanézete (\ref{subsec:user_index} alszakasz)}} \\
		\hline
		\multicolumn{2}{ | l | }{AS A} & adminisztrátor \\
		\hline
		\multicolumn{2}{ | l | }{I WANT TO} & a felhasználói fiókok listanézetét megtekinteni \\
		\hline
		\multicolumn{2}{ | l | }{SO THAT} & azokat listanézetben böngésszem, szerkesszem és/vagy töröljem \\
		\hline
		\multirow{3}{*}{1} & GIVEN & a fiókok listanézetének oldalán vagyok \\
		\cline{2-3}
		& WHEN & a szűrési feltételeket megadva a "Keresés" gombra kattintok \\
		\cline{2-3}
		& THEN & a listát az azoknak megfelelő fiókokkal az ott megadott oldalméretben kapom vissza \\
		\hline
		\multirow{3}{*}{2} & GIVEN & a fiókok listanézetének oldalán vagyok \\
		\cline{2-3}
		& WHEN & a "Vissza a teljes listához" gombra kattintok \\
		\cline{2-3}
		& THEN & a listát alapértelmezett (100 fiók/oldal) keresési feltételeivel kapom vissza \\
		\hline
		\multirow{3}{*}{3} & GIVEN & a fiókok listanézetének oldalán vagyok \\
		\cline{2-3}
		& WHEN & egy oldalszámra vagy a \faIcon{angle-double-left}, \faIcon{angle-left}, \faIcon{angle-right}, \faIcon{angle-double-right} ikonok valamelyikére kattintok \\
		\cline{2-3} 
		& THEN & az adott számú, vagy az első, előző, következő, illetve utolsó oldalra leszek navigálva \\
		\hline
		\multirow{3}{*}{4} & GIVEN & a fiókok listanézetének oldalán vagyok \\
		\cline{2-3}
		& WHEN & az egyik fiók (sor) jobb végén található \faIcon{info-circle} gombra kattintok \\
		\cline{2-3}
		& THEN & annak profiloldalára leszek navigálva \\
		\hline
		\multirow{3}{*}{5} & GIVEN & a fiókok listanézetének oldalán vagyok \\
		\cline{2-3}
		& WHEN & az egyik fiók (sor) jobb végén található \faIcon{edit} gombra kattintok \\
		\cline{2-3}
		& THEN & annak szerkesztési űrlapjára leszek navigálva \\
		\hline
		\multirow{3}{*}{6} & GIVEN & a fiókok listanézetének oldalán vagyok \\
		\cline{2-3}
		& WHEN & az egyik fiók (sor) jobb végén található \faIcon{trash} gombra kattintok \\
		\cline{2-3}
		& THEN & annak törlésének megerősítő kérdésének űrlapjára leszek navigálva \\
		\hline
	\end{longtable}
	\label{tab:sim_user_index}
\end{center}

\begin{note}
	Az alábbi felhasználói esetek lényegében megegyeznek a saját profil szerkesztésénél tárgyaltakkal. Adminisztrátorként viszont lehetőség van más profiljának módosítására, illetve ha a szerkesztés alatt álló profil nem saját, akkor akár még annak szerepkörének megváltoztatására is.
\end{note}
\begin{center}
	\centering
	\begin{longtable}{ | m{0.025\textwidth}| m{0.1\textwidth} | m{0.75\textwidth} | }
		\hline
		\multicolumn{3}{ | c | }{\textbf{Felhasználói fiók szerkesztése (\ref{subsec:user_edit} alszakasz)}} \\
		\hline
		\multicolumn{2}{ | l | }{AS A} & adminisztrátor \\
		\hline
		\multicolumn{2}{ | l | }{I WANT TO} & egy felhasználói fiókot szerkeszteni \\
		\hline
		\multicolumn{2}{ | l | }{SO THAT} & annak adatait frissítsem és/vagy szerepkörét megváltoztassam \\
		\hline
	\end{longtable}
	\label{tab:sim_admin_user_edit}
\end{center}

\begin{center}
	\centering
	\begin{longtable}{ | m{0.025\textwidth}| m{0.1\textwidth} | m{0.75\textwidth} | }
		\hline
		\multicolumn{3}{ | c | }{\textbf{Felhasználói fiók törlése} (\ref{subsec:user_delete} alszakasz)} \\
		\hline
		\multicolumn{2}{ | l | }{AS A} & adminisztrátor \\
		\hline
		\multicolumn{2}{ | l | }{I WANT TO} & valamelyik aloldalra navigálni \\
		\hline
		\multicolumn{2}{ | l | }{SO THAT} & azt az aloldalt elérjem \\
		\hline
		\multirow{3}{*}{1} & GIVEN & a fiók törlésének megerősítő kérdésének űrlapján vagyok \\
		\cline{2-3}
		& WHEN & a "Törlés" gombra kattintok \\
		\cline{2-3}
		& THEN & az előző oldalra leszek navigálva és a fiók törlésre kerül \\
		\hline
		\multirow{3}{*}{2} & GIVEN & a fiók törlésének megerősítő kérdésének űrlapján vagyok \\
		\cline{2-3}
		& WHEN & a "Vissza" gombra kattintok \\
		\cline{2-3}
		& THEN & az előző oldalra leszek navigálva \\
		\hline
		\end{longtable}
	\label{tab:sim_user_delete}
\end{center}
